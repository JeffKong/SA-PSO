\documentclass{beamer}
\usetheme{Madrid}
\usepackage[heading]{ctex}

%	TITLE PAGE
\title[PSO-SA] %optional
{基于~SA~的~PSO~算法}

\subtitle{期末展示}

\author[Zhe Kong] % (optional, for multiple authors)
{孔~哲\inst{1}}

\institute[IMUT] % (optional)
{
	\inst{1}%
	内蒙古工业大学~电力学院
}

\date[Dec. 2020] % (optional)
{2021年12月}

\logo{\includegraphics[height=1cm]{logo.png}}

%	Highlight the title of the current section
\AtBeginSection[]
{
	\begin{frame}
		\frametitle{目录}
		\tableofcontents[currentsection]
	\end{frame}
}

\setbeamertemplate{bibliography item}[text]

\begin{document}
	
	% insert title page---------------------------
	\begin{frame}
		\titlepage % Print the title page as the first slide
	\end{frame}

	%insert contents------------------------------
	\begin{frame}
		\frametitle{目录}
		\tableofcontents
	\end{frame}

	\section{引言}
	
	\begin{frame}
		\frametitle{引言}
		~~~~~~模拟退火(SA)是一种随机优化方法,广泛应用于组合优化问题。PSO-SA算法将模拟退火思想引入粒子群优化算法中,使可行解既能以较快速度收敛,又能有效避免搜索过程陷入局部最优解。\cite{b1}本展示采用Python实现PSO-SA算法,并与学习因子同步改变粒子群算法的结果进行了比较。\\~\\
		展示目标
		\begin{itemize}
			\item 介绍模拟退火算法
			\item 介绍模拟退火的粒子群算法
			\item 使用PSO-SA算法求解简单优化问题
		\end{itemize}
	\end{frame}

	\section{模拟退火算法思想}
	
	\begin{frame}
		\frametitle{模拟退火算法思想}
		\begin{block}{模拟退火算法}
			~~~~~~模拟退火算法的思想来源于对固体降温过程的模拟。\cite{b2}
		\end{block}
		~~~~~~将固体加热至一定温度,再让其冷却。固体加热时,内能增大,其内部粒子的热运动不断增强。随着温度的不断升高,粒子运动状态趋近于无序。冷却时,粒子运动状态逐渐趋于有序,在某个温度下可以达到平衡状态,最后在常温下达到基态,同时内能也趋近于某一定值。\cite{b1,b2}
	\end{frame}

	\begin{frame}
		\frametitle{模拟退火算法思想}
		~~~~~~算法将内能取为目标函数值$F$,将温度取为控制参数$T$,从一给定解$X$开始,在其邻域内随机产生一个新可行解$X~'$,使用Metropolis接受准则判断,算法持续进行“产生新解——计算目标函数差$\Delta F$——判断是否接受新解——接受或舍弃”的迭代过程,对应着固体在某一恒定温度$T$下趋于热平衡的过程。\\~\\
		~~~~~~经过多次迭代,可以求得给定控制参数$T$下最优解。然后减小控制参数$T$的值。重复执行上述迭代过程,当控制参数$T$逐渐减小并趋于零时,$X$也逐渐收敛于$X^{~*}$,对应于优化问题的整体最优解。
	\end{frame}

	\section{模拟退火算法步骤}
	
	\begin{frame}
		\frametitle{模拟退火算法步骤}
		算法步骤如下:
		\begin{enumerate}
			\item 初始化退火温度$T^{~k}$($k = 0$),产生随机初始解$X^{~0}$
			\item 进行如下迭代,得到温度$T^{~k}$下的新可行解$X^{~k}$
			\begin{itemize}
				\item 在$X$的领域中产生新的可行解$X~'$
				\item 计算目标函数值$F(X~')$和目标函数$F(X)$的差值$\Delta F$
				\item 若满足$min \{1, exp(- \Delta F / T^{~k}) \} > random [0, 1]$则接收$X~'$,否则舍弃
			\end{itemize}
			\item $T^{~k+1} =C \cdot T^{~k}, k = k + 1$,其中$C \in (0, 1)$\\若$X$满足收敛判据,则退火过程结束;否则,转步骤2继续迭代
		\end{enumerate}
		温度下降控制着迭代过程向优化方向进行,同时以概率$exp(- \Delta F / T^{~k})$接收劣质解,因此可以跳出局部极值点,只要初始温度足够高,迭代次数足够多,就能收敛到全局最优解。
	\end{frame}
	
	\section{模拟退火的粒子群算法}
	\begin{frame}
		\begin{block}{粒子群算法}
			~~~~~~粒子群算法将每个可能的解视为具有位置和速度的粒子。\cite{b2}
		\end{block}
		~~~~~~算法引入了速度这一概念,可以使目标函数加速收敛到最小值。在每次迭代中,每个粒子加速都向最佳位置趋近,每次加速都由随机数加权。\cite{b3}各粒子速度和位置更新公式如下:
		$$x^{~(i)} = x^{~(i)} + v^{~(i)}$$
		$$v^{~(i)} = w~v^{~(i)} + c_1 r_1 (x^{~(i)}_{best} − x^{~(i)}) + c_2 r_2 (x_{best} − x^{~(i)})$$
	\end{frame}
	\section{程序分析与展示}
	
	\begin{frame}
		\begin{enumerate}
			\item 初始化粒子的位置和速度
			\item 计算种群中每个粒子的目标函数值
			\item 更新各粒子的$x^{~(i)}_{best}$和$x_{best}$
			\item 重复执行下列步骤
			\begin{itemize}
				\item 对粒子的$x^{~(i)}_{best}$进行SA搜索
				\item 更新各粒子的$x^{~(i)}_{best}$
				\item 进行最优选择,更新粒子群的$x_{best}$
				\item 判断$x_{best}$是否满足算法终止条件。若是,转步骤5,否则转步骤4继续迭代
			\end{itemize}
			\item 输出全局最优解$x^{~*}$和目标函数值$F(x^{~*})$
		\end{enumerate}
	\end{frame}

	\section{参考文献}
	
	%a reference frame
	\begin{frame}
		\frametitle{参考文献}
		
		\begin{thebibliography}{99}
		\bibitem{b1}余胜威. MATLAB优化算法案例分析与应用[M]. 清华大学出版社, 2014.
			
		\bibitem{b2}Rao S S . Engineering Optimization: Theory and Practice: Fifth Edition[M]. John Wiley \& Sons, 2020.
			
		\bibitem{b3}Kochenderfer M J , Wheeler T A .Algorithms for Optimization[M]. The MIT Press, 2019.
			
		\end{thebibliography}
	\end{frame}

	% Insert a thank your frame ------------------------------------------------
	\begin{frame}
		\Huge{\centerline{展示结束}}
	\end{frame}
	
\end{document}